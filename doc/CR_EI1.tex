\pdfminorversion=4
\documentclass[]{article}
\usepackage[utf8]{inputenc}
\usepackage{amssymb,latexsym,amsmath}
\usepackage[a4paper,top=3cm,bottom=2cm,left=3cm,right=3cm,marginparwidth=1.75cm]{geometry}
\usepackage{graphicx}
\usepackage[colorlinks=true, allcolors=blue]{hyperref}
\begin{document}

% =====================================================================================
% Document : rendu du DM1
% Auteur : Xavier Gandibleux
% Année académique : 2020-2021

\section*{Livrable de l'exercice d'implémentation  1 : \\ Heuristiques de construction et d'amélioration gloutonnes}

%
% -----------------------------------------------------------------------------------------------------------------------------------------------------
%

\vspace{5mm}
\noindent
\fbox{
  \begin{minipage}{0.97 \textwidth}
    \begin{center}
      \vspace{1mm}
      \Large{Formulation du SPP}
      \vspace{1mm}
    \end{center}
  \end{minipage}
}
\vspace{2mm}

\noindent
Présenter la formulation du SPP. %Rechercher et citer 1 situation pratique que modélise le SPP en illustrant.

%
% -----------------------------------------------------------------------------------------------------------------------------------------------------
%

\vspace{5mm}
\noindent
\fbox{
  \begin{minipage}{0.97 \textwidth}
    \begin{center}
      \vspace{1mm}
        \Large{Modélisation JuMP (ou GMP) du SPP}
      \vspace{1mm}
    \end{center}
  \end{minipage}
}
\vspace{2mm}

\noindent
Présenter la modélisation JuMP (ou GMP) du SPP.

%
% -----------------------------------------------------------------------------------------------------------------------------------------------------
%

\vspace{5mm}
\noindent
\fbox{
  \begin{minipage}{0.97 \textwidth}
    \begin{center}
      \vspace{1mm}
        \Large{Instances numériques de SPP}
      \vspace{1mm}
    \end{center}
  \end{minipage}
}
\vspace{2mm}

\noindent
\begin{table}[h!]
\centering
\caption{Voici les 10 instances.}
\label{tab:instances_list} % Label modifié pour éviter les doublons
\begin{tabular}{l} % Une seule colonne
\toprule
\textbf{Instance} \\
\midrule
\texttt{pb\_100rnd0100.dat} \\
\texttt{pb\_200rnd0100.dat} \\
\texttt{pb\_500rnd0300.dat} \\
\texttt{pb\_500rnd1500.dat} \\
\texttt{pb\_500rnd1700.dat} \\
\texttt{pb\_200rnd0400.dat} \\
\texttt{pb\_200rnd0700.dat} \\
\texttt{pb\_1000rnd0100.dat} \\
\texttt{pb\_1000rnd0300.dat} \\
\texttt{didactic.dat} \\
\bottomrule
\end{tabular}
\end{table}





%
% -----------------------------------------------------------------------------------------------------------------------------------------------------
%

\vspace{5mm}
\noindent
\fbox{
  \begin{minipage}{0.97 \textwidth}
    \begin{center}
      \vspace{1mm}
        \Large{Heuristique de construction appliquée au SPP}
      \vspace{1mm}
    \end{center}
  \end{minipage}
}
\vspace{2mm}

\noindent
Présenter l'algorithme mis en \oe uvre. Illustrer sur un exemple didactique.

\subsubsection*{Mon heuristique de construction gloutonne (construction\_gloutonne)}

L'heuristique procède comme suit :
\begin{enumerate}
    \item Elle calcule d'abord l'utilité de chaque élément (colonne) avec sa fonction d'utilité~:
    $$ u = \frac{\text{coût}}{\text{nombre de contraintes couvertes}} $$
    \item Ensuite, elle trie les éléments par ordre décroissant d'utilité.
    \item Puis, elle parcourt la liste des éléments triés et ajoute chaque élément à la solution si cela ne crée pas de conflit avec les éléments déjà sélectionnés (c'est-à-dire, si aucune des contraintes couvertes par cet élément n'est déjà couverte par la solution partielle).
\end{enumerate}

\subsubsection*{Illustration sur \texttt{didactic.dat}}

\noindent\textbf{Étape 1 \& 2 : Calcul et Tri des Utilités}
\begin{itemize}
    \item Col 6 (Coût=13, Lignes=2) $\rightarrow$ Utilité = 6.5
    \item Col 7 (Coût=11, Lignes=2) $\rightarrow$ Utilité = 5.5
    \item Col 1 (Coût=10, Lignes=3) $\rightarrow$ Utilité = 3.33
    \item Col 4 (Coût=6, Lignes=2) $\rightarrow$ Utilité = 3.0
    \item Col 5 (Coût=9, Lignes=4) $\rightarrow$ Utilité = 2.25
    \item Col 3 (Coût=8, Lignes=4) $\rightarrow$ Utilité = 2.0
    \item Col 9 (Coût=6, Lignes=4) $\rightarrow$ Utilité = 1.5
    \item Col 2 (Coût=5, Lignes=4) $\rightarrow$ Utilité = 1.25
    \item Col 8 (Coût=4, Lignes=4) $\rightarrow$ Utilité = 1.0
\end{itemize}

\noindent\textbf{Étape 3 : Construction}

\noindent Solution initiale = [], Lignes couvertes = []
\begin{itemize}
    \item \textbf{Évaluation Col 6} (utilité 6.5)...
    \begin{itemize}[label=--]
        \item Action: \textbf{ACCEPTÉE}. (Pas de conflit)
        \item Solution = [6]
        \item Lignes couvertes par Col 6: [3, 5]
        \item Lignes couvertes (Total) = [3, 5]
    \end{itemize}
    \item \textbf{Évaluation Col 7} (utilité 5.5)...
    \begin{itemize}[label=--]
        \item Action: \textbf{ACCEPTÉE}. (Pas de conflit)
        \item Solution = [6, 7]
        \item Lignes couvertes par Col 7: [1, 6]
        \item Lignes couvertes (Total) = [1, 3, 5, 6]
    \end{itemize}
    \item \textbf{Évaluation Col 1} (utilité 3.33)...
    \begin{itemize}[label=--]
        \item Action: \textbf{REJETÉE}. (Conflit sur Ligne 1)
    \end{itemize}
    \item \textbf{Évaluation Col 4} (utilité 3.0)...
    \begin{itemize}[label=--]
        \item Action: \textbf{ACCEPTÉE}. (Pas de conflit)
        \item Solution = [6, 7, 4]
        \item Lignes couvertes par Col 4: [4, 7]
        \item Lignes couvertes (Total) = [1, 3, 4, 5, 6, 7]
    \end{itemize}
    \item \textbf{Évaluation Col 5} (utilité 2.25)...
    \begin{itemize}[label=--]
        \item Action: \textbf{REJETÉE}. (Conflit sur Ligne 1)
    \end{itemize}
    \item \textbf{Évaluation Col 3} (utilité 2.0)...
    \begin{itemize}[label=--]
        \item Action: \textbf{REJETÉE}. (Conflit sur Ligne 1)
    \end{itemize}
    \item \textbf{Évaluation Col 9} (utilité 1.5)...
    \begin{itemize}[label=--]
        \item Action: \textbf{REJETÉE}. (Conflit sur Ligne 3)
    \end{itemize}
    \item \textbf{Évaluation Col 2} (utilité 1.25)...
    \begin{itemize}[label=--]
        \item Action: \textbf{REJETÉE}. (Conflit sur Ligne 1)
    \end{itemize}
    \item \textbf{Évaluation Col 8} (utilité 1.0)...
    \begin{itemize}[label=--]
        \item Action: \textbf{REJETÉE}. (Conflit sur Ligne 1)
    \end{itemize}
\end{itemize}

\vspace{3mm}
\noindent \textbf{Solution gloutonne finale:} [6, 7, 4] \\
\textbf{Valeur ($z_{\text{glouton}}$):} 30

%
% -----------------------------------------------------------------------------------------------------------------------------------------------------
%

\vspace{5mm}
\noindent
\fbox{
  \begin{minipage}{0.97 \textwidth}
    \begin{center}
      \vspace{1mm}
        \Large{Heuristique d'amélioration appliquée au SPP}
      \vspace{1mm}
    \end{center}
  \end{minipage}
}
\vspace{2mm}

\noindent
Présenter l'algorithme mis en oeuvre. Illustrer sur un exemple didactique.


\subsubsection*{Mon heuristique d'amélioration (descente\_local)}

Mon heuristique d'amélioration est une descente locale, \texttt{descente\_local}, qui fonctionne comme ceci~:
L'algorithme commence à partir d'une solution initiale (ici, la solution gloutonne).

\begin{enumerate}
    \item \textbf{Génération de voisinage~:} À chaque itération, la fonction appelle la fonction de voisinage (\texttt{generer\_voisinage\_1\_1}). Celle-ci génère l'ensemble de tous les voisins valides pouvant être atteints en effectuant un échange k-p de type 1-1 (retirer un élément de la solution et en ajouter un qui n'y est pas, tout en respectant les contraintes).

    \item \textbf{Évaluation et sélection~:} On évalue le coût de chaque voisin généré pour trouver le meilleur voisin (celui avec le coût le plus élevé).
    \begin{itemize}
        \item \textbf{Si} le coût de ce meilleur voisin est supérieur au coût de la solution courante, on met à jour la solution courante avec ce meilleur voisin et on répète le processus (retour à l'étape 1).
        \item \textbf{Sinon}, si aucun voisin n'améliore la solution courante, l'algorithme s'arrête et retourne la solution courante comme optimum local.
    \end{itemize}
\end{enumerate}

\subsubsection*{Illustration sur \texttt{didactic.dat}}

\noindent \textbf{Itération de Descente n°1}
\begin{itemize}
    \item Solution courante: [6, 7, 4] (Coût = 30)
    \item Génération de 0 voisins (1-1) valides...
    \item \textbf{Résultat~:} PAS D'AMÉLIORATION. Optimum local atteint.
\end{itemize}

\paragraph{Analyse de l'illustration}
Ici, pendant la génération des voisins, aucun voisin valide n'a pu être généré à partir de la solution initiale gloutonne [6, 7, 4]. L'algorithme s'est donc arrêté immédiatement et a retourné la solution initiale comme solution finale. Notre solution était déjà un optimum local pour ce voisinage.


%
% -----------------------------------------------------------------------------------------------------------------------------------------------------
%

\vspace{5mm}
\noindent
\fbox{
  \begin{minipage}{0.97 \textwidth}
    \begin{center}
      \vspace{1mm}
        \Large{Expérimentation numérique}
      \vspace{1mm}
    \end{center}
  \end{minipage}
}
\vspace{2mm}

\noindent
Présenter l'environnement machine sur lequel les algorithmes vont tourner (référence). 
Présenter  sous forme synthétique (tableau, graphique...) les résultats obtenus pour les 10 instances sélectionnées.


\noindent
\textbf{Environnement machine :}
Les algorithmes ont tourné sur la machine suivante :
\begin{itemize}
    \item \textbf{Système d'exploitation :} Windows 11
    \item \textbf{Processeur :} AMD RYZEN 7 5700X 8-Core
    \item \textbf{Mémoire :} 16Go
\end{itemize}

\noindent
\begin{table}[h!] 
\centering
\caption{Résultats des heuristiques gloutonne et locale sur 10 instances SPP.}
\label{tab:resultats_spp}
\begin{tabular}{lrrrc}
\toprule
\textbf{Instance} & \textbf{$z_{\text{glouton}}$} & \textbf{$t_{\text{glouton}}$ (s)} & \textbf{$z_{\text{local}}$} & \textbf{$t_{\text{local}}$ (s)} \\
\midrule
\texttt{pb\_100rnd0100.dat} & 342.0 & 0.000 & 343.0 & 0.001 \\
\texttt{pb\_200rnd0100.dat} & 351.0 & 0.001 & 365.0 & 0.008 \\
\texttt{pb\_500rnd0300.dat} & 674.0 & 0.004 & 693.0 & 0.046 \\
\texttt{pb\_500rnd1500.dat} & 1059.0 & 0.009 & 1115.0 & 0.136 \\
\texttt{pb\_500rnd1700.dat} & 150.0 & 0.004 & 154.0 & 0.003 \\
\texttt{pb\_200rnd0400.dat} & 55.0 & 0.000 & 55.0 & 0.004 \\
\texttt{pb\_200rnd0700.dat} & 945.0 & 0.000 & 955.0 & 0.004 \\
\texttt{pb\_1000rnd0100.dat} & 49.0 & 0.192 & 49.0 & 0.001 \\
\texttt{pb\_1000rnd0300.dat} & 507.0 & 0.023 & 528.0 & 0.119 \\
\texttt{didactic.dat} & 30.0 & 0.000 & 30.0 & 0.000 \\
\bottomrule
\end{tabular}
\end{table}



%
% -----------------------------------------------------------------------------------------------------------------------------------------------------
%

\vspace{5mm}
\noindent
\fbox{
  \begin{minipage}{0.97 \textwidth}
    \begin{center}
      \vspace{1mm}
        \Large{Discussion}
      \vspace{1mm}
    \end{center}
  \end{minipage}
}
\vspace{2mm}

\noindent
Questions type pour mener votre discussion :

\begin{itemize}
\item au regard des temps de résolution requis par le solveur MIP (GLPK) pour obtenir une solution optimale à  l'instance considérée, l'usage d'une heuristique se justifie-t-il?

\item avec pour référence la solution optimale, quelle est la qualité des solutions obtenues avec l'heuristique de construction et l'heuristique d'amélioration? \\
Sur le plan des temps de résolution, quel est le rapport  entre le temps consommé par le solveur MIP et vos heuristiques?

\item Le recours aux (méta)heuristiques apparaît-il prometteur ? \\
Entrevoyez-vous des pistes d'amélioration à apporter à vos heuristiques?

\vfill
\break

\end{itemize}


\end{document}